% REMEMBER: You must not plagiarise anything in your report. Be extremely careful.

\documentclass{l4proj}

    
%
% put any additional packages here
%

\begin{document}

%==============================================================================
%% METADATA
\title{HarmonyScape: Inclusive 3D Exploration with Music for Intellectual Disabilities}
\author{Robbie Tippen}
\date{March 22, 2024}

\maketitle

%==============================================================================
%% ABSTRACT
\begin{abstract}
    
    This dissertation investigates the development and implementation of HarmonyScape, an inclusive initiative tailored for individuals with intellectual disabilities, including Alzheimer's. Inspired by the profound link between music and cognitive function, HarmonyScape creates a captivating 3D world where music dynamically responds to user exploration. Through character control in an open-world setting and interactive music-based mini games, players unlock new instruments, filling out the games' musical landscape. The project's aim is to provide a multi-sensory environment that potentially aids cognitive well-being while offering an engaging, interactive experience. Initial user feedback indicates high levels of engagement and enjoyment, suggesting promising avenues for inclusive gaming and therapeutic interventions. This research underscores the significance of tailored approaches to accessibility and engagement in fostering cognitive well-being among individuals with intellectual disabilities.
    \vskip 0.5em
    Every abstract follows a similar pattern. Motivate; set aims; describe work; explain results.
    \vskip 0.5em
    ``XYZ is bad. This project investigated ABC to determine if it was better. 
    ABC used XXX and YYY to implement ZZZ. This is particularly interesting as XXX and YYY have
    never been used together. It was found that  
    ABC was 20\% better than XYZ, though it caused rabies in half of subjects.''
\end{abstract}

\begin{abstract}
    \textbf{Inclusive 3D Exploration with Music for Intellectual Disabilities}

 

    Imagine an inclusive initiative tailored specifically for individuals with intellectual disabilities, including conditions like Alzheimer's. Through Unity, we plan to create a captivating 3D world that generates music in response to exploration. This project is inspired by the unique relationship between music and cognitive function (Baird \& Samson, 2015), and its therapeutic potential for those with dementia (Vink et al., 2003). By immersing users in a visually engaging environment where music dynamically responds to movement, we aim to provide a multisensory experience that not only captivates but also potentially aids cognitive well-being in an enjoyable, interactive manner.

 

    \textbf{Reading}  
    
    Baird, A., \& Samson, S. (2015). Music and dementia. Progress in brain research, 217, 207-235.
    
     
    
    Vink, A. C., Bruinsma, M. S., \& Scholten, R. J. (2003). Music therapy for people with dementia. Cochrane database of systematic reviews, (4).
\end{abstract}

%==============================================================================

% EDUCATION REUSE CONSENT FORM
% If you consent to your project being shown to future students for educational purposes
% then insert your name and the date below to  sign the education use form that appears in the front of the document. 
% You must explicitly give consent if you wish to do so.
% If you sign, your project may be included in the Hall of Fame if it scores particularly highly.
%
% Please note that you are under no obligation to sign 
% this declaration, but doing so would help future students.
%
\def\consentname {Robbie Tippen} % your full name
\def\consentdate {27 February 2024} % the date you agree
%
\educationalconsent


%==============================================================================
\tableofcontents

%==============================================================================
%% Notes on formatting
%==============================================================================
% The first page, abstract and table of contents are numbered using Roman numerals and are not
% included in the page count. 
%
% From now on pages are numbered
% using Arabic numerals. Therefore, immediately after the first call to \chapter we need the call
% \pagenumbering{arabic} and this should be called once only in the document. 
%
% Do not alter the bibliography style.
%
% The first Chapter should then be on page 1. You are allowed 40 pages for a 40 credit project and 30 pages for a 
% 20 credit report. This includes everything numbered in Arabic numerals (excluding front matter) up
% to but excluding the appendices and bibliography.
%
% You must not alter text size (it is currently 10pt) or alter margins or spacing.
%
%
%==================================================================================================================================
%
% IMPORTANT
% The chapter headings here are **suggestions**. You don't have to follow this model if
% it doesn't fit your project. Every project should have an introduction and conclusion,
% however. 
%
%==================================================================================================================================
\chapter{Introduction}

% reset page numbering. Don't remove this!
\pagenumbering{arabic} 

% The project is inspired by research indicating the therapeutic potential of music for individuals with dementia, as highlighted by studies such as Vink et al. (2003). By leveraging the relationship between music and cognitive function, the initiative seeks to explore how interactive music experiences can benefit individuals with intellectual disabilities.

TOODO: \begin{itemize}
    \item Aims
\end{itemize}


This opening chapter sets the stage for the HarmonyScape project by introducing the motivations and goals that led to its development. The following sections of this paper will explore distinct segments of the project, covering its background, design, implementation, and evaluation.

\section{Motivation}
As the prevalence of dementia increases within our aging population, with the number of older people with dementia in the UK projected to rise by 80\% from almost 885,000 in 2019 to around 1.6 million in 2040 (Wittenberg \emp{et al.} 2019), the development of effective treatments and engaging activities to manage its symptoms becomes imperative.  The means of music as a form of therapy in the treatment of intellectual disabilities, such as Alzheimer's, is a unique approach that taps into preserved cognitive and emotional functions in individuals with dementia.

Research by Vink AC \emph{et al.} (2004) shows promising results that for individuals with dementia, music-based therapeutic interventions can be employed to target depressive symptoms. While the potential benefits of game-based activities for cognitive function in dementia are highlighted by Vilkki et al. (2019), combining these approaches in a therapeutic setting has been seldom explored. This suggests a valuable avenue for future research in dementia care.

This combined approach is particularly promising for our aging population. As lifespans increase, so does the prevalence of dementia. However, this generation also coincides with a rise in video game popularity. People who were 14 years old in 1973, the year the first home video game console, the Magnavox Odyssey\footnote{https://www.computinghistory.org.uk/det/16909/Magnavox-Odyssey/}, was released globally, would be 65 today. This being the age where the risk of dementia significantly increases (Chen \emph{et al.,} 2009). This generation's familiarity with games can be leveraged to create engaging and effective therapeutic activities. Despite the clear potential of this promising avenue for leveraging a familiar and potentially engaging activity, it remains largely unexplored.

- Actively making music engages those with Alzheimer's

- multisensory experiences provide mental and emotional stimulation for those with alzhiemers

- enjoyment of music is something that doesn't dissipate as dementia progresses

\section{Aims}



% Why should the reader care about what are you doing and what are you actually doing?
% \section{Guidance}

% \textbf{Motivate} first, then state the general problem clearly. 

% \section{Writing guidance}
% \subsection{Who is the reader?}

% This is the key question for any writing. Your reader:

% \begin{itemize}
%     \item
%     is a trained computer scientist: \emph{don't explain basics}.
%     \item
%     has limited time: \emph{keep on topic}.
%     \item
%     has no idea why anyone would want to do this: \emph{motivate clearly}
%     \item
%     might not know \emph{anything} about your project in particular:
%     \emph{explain your project}.
%     \item
%     but might know precise details and check them: \emph{be precise and
%     strive for accuracy.}
%     \item
%     doesn't know or care about you: \emph{personal discussions are
%     irrelevant}.
% \end{itemize}

% Remember, you will be marked by your supervisor and one or more members
% of staff. You might also have your project read by a prize-awarding
% committee or possibly a future employer. Bear that in mind.

% \subsection{References and style guides}
% There are many style guides on good English writing. You don't need to
% read these, but they will improve how you write.

% \begin{itemize}
%     \item
%     \emph{How to write a great research paper} \cite{Pey17} (\textbf{recommended}, even though you aren't writing a research paper)
%     \item
%     \emph{How to Write with Style} \cite{Von80}. Short and easy to read. Available online.
%     \item
%     \emph{Style: The Basics of Clarity and Grace} \cite{Wil09} A very popular modern English style guide.
%     \item
%     \emph{Politics and the English Language} \cite{Orw68}  A famous essay on effective, clear writing in English.
%     \item
%     \emph{The Elements of Style} \cite{StrWhi07} Outdated, and American, but a classic.
%     \item
%     \emph{The Sense of Style} \cite{Pin15} Excellent, though quite in-depth.
% \end{itemize}

\subsubsection{Citation styles}

\begin{itemize}
\item If you are referring to a reference as a noun, then cite it as: ``\citet{Orw68} discusses the role of language in political thought.''
\item If you are referring implicitly to references, use: ``There are many good books on writing \citep{Orw68, Wil09, Pin15}.''
\end{itemize}

% There is a complete guide on good citation practice by Peter Coxhead available here: \url{http://www.cs.bham.ac.uk/~pxc/refs/index.html}. 
% If you are unsure about how to cite online sources, please see this guide: \url{https://student.unsw.edu.au/how-do-i-cite-electronic-sources}.

% \subsection{Plagiarism warning}

% \begin{highlight_title}{WARNING}
    
%     If you include material from other sources without full and correct attribution, you are commiting plagiarism. The penalties for plagiarism are severe.
%     Quote any included text and cite it correctly. Cite all images, figures, etc. clearly in the caption of the figure.
% \end{highlight_title}


%==================================================================================================================================
\chapter{Background}
To DO:
\begin{itemize}
    \item Beginning section
    \item ADD A SECTION ABOUT JARRING NOISES DISORIENTING USERS WITH DEMENTIA (Music Therapy in the Treatment of Dementia: A Review Article(Shirsat, Jha and Verma, 2023)) ((Caspar et al., 2017))
    \item Write Follow The Sound
    \item Move Minigame descriptions to Design
\end{itemize}


What did other people do, and how is it relevant to what you want to do?


% Azlzhiemers (could be weaved into introduction for the section)
% Music Theory in relation
% Related work (musical games targeted at alzhiemers)

\section{Alzheimer's disease}
Alzheimer's disease is a progressive neurological disorder that can affect memory, thinking skills and behaviour. As of 2024, it is the most common cause of dementia, a group of brain disorders that result in loss of intellectual and social skills that is severe enough to interfere with daily functioning (Li et al., 2024). Alzheimer's typically begins slowly and as the disease progresses, individuals can experience memory loss, confusion, difficulty with language and communication, impaired judgment, and changes in personality (Li et al., 2024). Currently, there is no cure for Alzheimer's disease, but treatments and interventions can manage symptoms and improve quality of life for those affected by the disease.

Both pharmacological and non-pharmacological treatments have been trialed to alleviate symptoms of dementia. Pharmacological treatments, like acetylcholinesterase inhibitors, primarily target cognitive symptoms without altering the progression of the disease.
These treatments have shown minimal effectiveness in easing the behavioral and psychological symptoms linked to these diseases (Dyer et al., 2018).

Conversely, non-pharmacological interventions offer supplementary treatments, presenting diverse strategies to enhance the well-being of individuals with dementia, reduce behavioral incidents, and enhance or maintain their quality of life (Dyer et al., 2018).

\section{Music Therapy}

Music therapy is defined by the World Federation of Music Therapy (2011) as the professional application of musical elements (sound, rhythm, melody, and harmony) by qualified therapists to facilitate communication, relationships, learning, expression, and other therapeutic objectives. It is employed across medical, educational, and everyday settings to optimise quality of life and enhance physical, social, communicative, emotional, intellectual, and spiritual health and well-being.

\emph{Music therapy for people with dementia} by Vink AC \emph{et al.} (2004) discusses music therapy as one potential non-pharmacological treatment for people with different forms of dementia, such as Alzheimer's. Of particular relevance to our aims are the discussions of the distinct varieties of music therapy, Vink AC \emph{et al.} (2004) distinguish two main types of music therapy.
\newline

\textbf{Receptive Music Therapy}
\newline

In receptive music therapy, individuals passively engage with music chosen or performed by the therapist. They may listen to live music, recordings, or the therapist's singing. This form of therapy focuses on the emotional and psychological responses evoked by the music. The therapist selects music tailored to the individual's needs, aiming to stimulate relaxation, emotional expression, or cognitive engagement (Vink AC \emph{et al.} 2004).
\newline

\textbf{Active Music Therapy}
\newline

Active music therapy involves active participation from the individual receiving therapy. This can include playing musical instruments, singing, improvisation, or engaging in structured music activities led by the therapist. Active participation encourages physical movement, cognitive stimulation, social interaction, and self-expression. It empowers individuals to actively engage with music-making processes, fostering a sense of agency and creativity in a therapeutic environment (Vink AC \emph{et al.} 2004).

\section{Related Works}
In my comprehensive review of existing literature, I found a notable absence of multisensory experiences tailored for individuals with intellectual disabilities such as Alzheimer’s that incorporate active music theory within a gaming setting. However, a range of experiences created for both academic and commercial purposes have been introduced with a similar goal as that of HarmonyScape in aiding those with intellectual disabilities, including conditions like Alzheimer’s, in enhancing their cognitive well-being in an enjoyable, interactive manner.

\subsection{AARP Staying Sharp}
AARP Staying Sharp is an online platform provided by AARP (formerly known as the American Association of Retired Persons) that offers a variety of brain-training games and activities designed to help individuals maintain cognitive function and mental acuity as they age. These games cover a range of cognitive skills such as memory, attention, problem-solving, and language.

\begin{figure}[h]
    \centering
    \includegraphics[width=0.8\linewidth]{images/AARP_Staying_Sharp.png}    

    \caption{AARP Staying Sharp's game library, showcasing the range of cognitive games and activities offered by AARP to support mental acuity and brain health.
    }

    % use the notation fig:name to cross reference a figure
    \label{fig:aarp} 
\end{figure}

Although AARP Staying Sharp has great potential in helping individuals maintain cognitive function through its variety of brain-training games and activities, a key downfall is the sole use of uni-sensory stimulation, and so is missing out on the range of potential benefits shown from multisensory treatment in those with neurodegenerative diseases, such as Alzheimer's.

Sánchez et al. (2016) discuss these potential positive effects in comparison to uni-sensory experiences in the treatment of people with severe dementia. Specifically, they discuss the beneficial impact on anxiety symptoms that comes from stimulating different senses which allows for greater sensory environment control. HarmonyScape aims to capitalise on the principles outlined by Sánchez et al. (2016), offering multisensory engagement that intertwines visual and auditory stimuli.


WRITE A DOWNFALL ABOUT HOW IT DOESNT EXPLICITY USE 

WRITE ABOUT A DOWNFALL FOR SOMETHING BEING THE EXPLICIT USE OF FAIL STATES

WRITE A DOWNFALL FOR SOMETHING BEING LACKING IN USER CREATIVITY WHICH CAN LIMIT ENGAGEMENT

\section{Minigames}
This section describes the different games that influenced the minigames in HarmonyScape, how they incorporate music theory and how they can be used as an interactive means of exercising cognitive skills for those with Alzheimer's and similar neurodegenerative diseases.

\subsection{Card Matching}
De Siqueira et al. (2017) discusses the advantages of utilising well-known traditional games in the development of games tailored for individuals with intellectual disabilities like Alzheimer's. Since many elderly individuals are likely familiar with the rules of these games, minimal introduction and assistance are required for them to be engaged (De Siqueira et al., 2017). Moreover, for those with Alzheimer's, playing these games can evoke memories of past experiences, such as playing as children or with grandchildren, thus facilitating a personalised approach that enhances engagement and motivation (De Siqueira et al., 2017). By incorporating personalisation, memory training games can be even more effective for Alzheimer's patients (Eichhorn et al., 2018).

This inspired the choice of a card matching game where the player must uncover pairs of cards by flipping over face-down cards on the playing field and remembering their positions. Furthermore, playing a card matching game requires attention, and memory recall, which can help stimulate cognitive function in individuals with dementia (Eichhorn et al., 2018). The game directly targets memory skills by challenging players to remember the location of matching pairs of cards, which Eichhorn et al. (2018) discuss as having the potential to improve short-term memory and concentration.

\begin{figure}[h]
    \centering
    \includegraphics[width=1.0\linewidth]{dissertation/images/mem_matching.jpg}    

    \caption{Screenshots of different scenes when playing the digital card matching memory game by MemoryMatching.com
    }

    \label{fig:mem_match} 
\end{figure}

Figure \ref{fig:mem_match} shows screenshots from a digital implementation of the card matching memory game by MemoryMatching.com. HarmonyScape implements this idea but in a 3D space where the player controlled character has to make contact with the cards in order to flip them.


\subsection{Sliding Puzzle}

A sliding puzzle is a classic game comprising a grid of square or rectangular tiles, with one space left empty to enable movement. Players aim to rearrange the tiles by sliding them horizontally or vertically into the vacant space until they align in a predetermined pattern or sequence, typically forming a complete picture or achieving a specific numerical order (Hearn, 2005). 

Sasaki et al. (2020) explored the use of sliding puzzle's in the treatment of dementia with their brain training application. Employing the same mechanism found in the traditional sliding block puzzle, the user manipulates the panel within a single frame, utilising available space to organise it according to the desired arrangement (Sasaki et al., 2020). The act of sliding the puzzle pieces activates the frontal lobe, responsible for memory, as the user anticipates the direction in which the moving panels can be adjusted to achieve the desired order (Hirono et al., 1997). The user is required to alter the positions of the panels within the sliding block puzzle while retaining the original layout in memory (Sasaki et al., 2020).

The spatial recognition capacity of the frontal lobe is engaged as the user formulates the steps necessary to reach the correct solution by comprehending the panel positions (Hirono et al., 1997). Completing the puzzle necessitates memory retention, figure arrangement, and spatial comprehension, suggesting a potential for preventing dementia progression (Sasaki et al., 2020).

\begin{figure}[h]
    \centering
    \begin{subfigure}[b]{0.45\textwidth}
        \includegraphics[width=\textwidth]{dissertation/images/uncompleted_sliding.png}
        \caption{Uncompleted sliding puzzle.}
        \label{fig:slide_uncompleted}
    \end{subfigure}
    ~ 
    \begin{subfigure}[b]{0.45\textwidth}
        \includegraphics[width=\textwidth]{dissertation/images/completed_sliding.png}
        \caption{Completed sliding puzzle.}
        \label{fig:slide_completed}
    \end{subfigure}
    ~  
    \caption{Screenshots from brain training application developed by Sasaki et al. (2020), showing incomplete
    }\label{fig:sliding_puzzle}
\end{figure}


Similar to the traditional sliding block puzzle and the implementation by Sasaki et al. (2020), in HarmonyScape the user manipulates the panel within the confines of one frame to arrange it into the desired order. Given that the sliding block puzzle requires goal-oriented memory, spatial awareness, and figure arrangement, research conducted by Sasaki et al. (2020) suggests that engagement with such puzzles by users may potentially suppress the progression of dementia.

\subsection{Music Memory}
The concept of memory games involving sequences of sounds can be traced back to various sources and inspirations. However, one notable origin of such games is the electronic game "Simon" which was invented by Ralph H. Baer and Howard J. Morrison and manufactured and distributed by Milton Bradley in 1978 (Baer \& Morrison, 1980). "Simon" consists of four colored buttons (green, red, blue, and yellow) that light up and produce unique sounds in a sequence (Baer \& Morrison, 1980). Players have to memorise the sequence and then repeat it back by pressing the corresponding buttons (Baer \& Morrison, 1980).

\begin{figure}[h]
    \centering
    \includegraphics[width=1.0\linewidth]{dissertation/images/simon.jpg}    

    \caption{Screenshots of online implementation of the game "Simon" from "freesimon.org"
    }

    \label{fig:simon} 
\end{figure}

Attention impairments in Alzheimer’s Disease represent a significant aspect of the cognitive decline observed in affected individuals, despite Alzheimer’s Disease being primarily recognized as a memory disorder (Hennawy et al., 2019).

\subsubsection{Selective Attention}
pertains to the cognitive capacity to focus on a singular stimulus while ignoring surrounding distractions (Chau et al., 2015). Impairments in selective attention exacerbate linearly with Alzheimer’s Disease severity (Chau et al., 2015).

\subsubsection{Sustained Attention} is differentiated from other classes of attention by the duration of the required activity (Fortenbaugh et al., 2017). More specifically, sustained attention is defined as the capacity to sustain focus on a particular task over an extended duration (Fortenbaugh et al., 2017).

A systematic review conducted by Clare et al. (2003) examined the effectiveness of cognitive interventions in dementia care. The review found evidence supporting the beneficial effects of cognitive stimulation interventions on various aspects of cognitive function, including attention and concentration. Activities such as "Simon" which necessitate both selective attention to focus on specific elements and sustained attention to accurately remember and reproduce sequences, have been identified as interventions associated with positive outcomes in cognitive function (Clare et al., 2003).
%Regular practice with games such as "Simon" may help individuals with Alzheimer's improve their ability to concentrate and sustain attention over time.


\subsection{Follow The Sound}


%==================================================================================================================================
\chapter{Requirements}

This chapter dives into the specific requirements HarmonyScape needs to meet to effectively achieve the goals outlined in Chapter 1. These requirements leverage and build upon the research and relevant concepts explored in Chapter 2, ensuring the game aligns with established best practices.

% \section{Guidance}
% Make it clear how you derived the constrained form of your problem via a clear and logical process.

\section{Functional Requirements}
This section presents the functional requirements for HarmonyScape, informed by the research in Chapter 2. The MoSCow method prioritises these requirements, categorising them based on their necessity to achieving the evaluation goals: some are essential for core functionality (Must Haves), while others significantly enhance the user experience (Should Haves), and some are desirable for future iterations (Could Haves).

\subsection{Must Have}
Interactive Music: HarmonyScape's music must be interactive, dynamically responding to the player's movements. This aligns with Active Music Theory (Vink AC \emph{et al.,} 2004), where the players active participation directly affects the games soundscape. The aim of this feature is to encourage a sense of control and creative exploration through allowing players to influence the musical composition. Whilst also facilitating cognitive stimulation through providing opportunity for players to solve problems or puzzles through musical interaction.

Multisensory Experience: HarmonyScape must incorporate a multisensory experience, specifically through the use of visual and auditory stimuli. This aligns with research by Sánchez et al. (2016) suggesting that stimulating multiple senses can benefit individuals living with dementia who are experiencing anxiety symptoms. This should empower players to exercise greater control over their sensory environment, potentially mitigating anxiety symptoms associated with dementia (Sánchez et al., 2016).

Gentle Audio Design: I intend to implement the integration of new instruments into the game's soundscape, based on player input, to further develop the concept of interactive music within the game. The following chapter, "Design," explores this concept further. Research by Shirsat, Jha and Verma (2023) shows that jarring or disorienting changes in tone can exacerbate anxiety and confusion in individuals with dementia. Therefore, HarmonyScape should prioritises a calm and predictable soundscape. Familiar and consistent music plays a vital role in promoting a sense of security and reducing potential disorientation for people with dementia (Caspar et al., 2017). The gentle melody should serve as a constant anchor, allowing players to focus on the game-play without auditory distractions.

\subsection{Should Have}
Progress Tracker: HarmonyScape should include a progress tracker within the user interface. This tracker's purpose is to provide clear and consistent visual feedback on instrument collection throughout game-play. This tracker should utilise a simple and clear visual representation, such as colour-filling elements, to provide immediate feedback on instrument collection. Players with dementia often experience short-term memory challenges. A progress tracker could be particularly helpful for this group. Seeing progress can reinforce a sense of accomplishment and provides a clear marker of progress within the game (Vermeir et al., 2020). This, in turn, could enhance motivation and engagement for players living with dementia as they progress through HarmonyScape.

\subsection{Could Have}
Incorporation of reminiscence therapy: HarmonyScape could be further developed by incorporating elements of reminiscence therapy. This approach utilises familiar music to evoke positive memories and emotional well-being in users (Bhar and Sunil, 2014).  Players could be given the option to choose songs, familiar to them, they want to incorporate into the game. Research done in reminiscence therapy indicates that familiar music can trigger personal memories and enhance emotional well-being in those living with dementia (Cuevas et al., 2020). Integrating song selection could potentially unlock positive memories associated with their chosen music, expanding the potential therapeutic benefits of HarmonyScape.

\section{Non-functional Requirements}
HarmonyScape's non-functional requirements address broader aspects that contribute to a positive user experience during evaluation. As with the functional requirements, prioritisation has been applied using the MoSCoW method.

\subsection{Must Have}
Accessibility: I aim to ensure clear and concise instructions to support players with varying cognitive abilities. This can be achieved through visuals and audio cues that are simple and unambiguous, such as written instructions and contextual audio prompts.

\subsection{Should Have}
Visible Interaction: Making it obvious that player input is affecting the game. For example, upon player input, subtle animations for the instruments themselves and their corresponding game-play. A brief visual flourish or change in appearance could provide immediate confirmation that the player's action has registered. This type of clear and consistent visual feedback can be particularly beneficial for players with dementia, as studies suggest it can improve engagement and reduce frustration in users with cognitive decline (Ballard et al., 2008).


%==================================================================================================================================
\chapter{Design}
How is this problem to be approached, without reference to specific implementation details? 

TODO:
\begin{itemize}
    \item Overview
    \item Player Movement
    \item Music dynamically changing with movement
    \item Minigames
\end{itemize}


\section{Guidance}
Design should cover the abstract design in such a way that someone else might be able to do what you did, but with a different language or library or tool.

%==================================================================================================================================
\chapter{Implementation}

TODO:
\begin{itemize}
    \item How the player movement script works
    \item How the dynamic music was implemented
    \item How each minigame was implementd
    \begin{itemize}
        \item Card Matching
        \item Sliding Puzzle
        \item Music Memory
        \item Follow The Sound
    \end{itemize}
\end{itemize}

What did you do to implement this idea, and what technical achievements did you make?
\section{Guidance}
You can't talk about everything. Cover the high level first, then cover important, relevant or impressive details.



\section{General points}

These points apply to the whole dissertation, not just this chapter.



\subsection{Figures}
\emph{Always} refer to figures included, like Figure \ref{fig:relu}, in the body of the text. Include full, explanatory captions and make sure the figures look good on the page.
You may include multiple figures in one float, as in Figure \ref{fig:synthetic}, using \texttt{subcaption}, which is enabled in the template.



% Figures are important. Use them well.
\begin{figure}
    \centering
    \includegraphics[width=0.5\linewidth]{images/relu.pdf}    

    \caption{In figure captions, explain what the reader is looking at: ``A schematic of the rectifying linear unit, where $a$ is the output amplitude,
    $d$ is a configurable dead-zone, and $Z_j$ is the input signal'', as well as why the reader is looking at this: 
    ``It is notable that there is no activation \emph{at all} below 0, which explains our initial results.'' 
    \textbf{Use vector image formats (.pdf) where possible}. Size figures appropriately, and do not make them over-large or too small to read.
    }

    % use the notation fig:name to cross reference a figure
    \label{fig:relu} 
\end{figure}


\begin{figure}
    \centering
    \begin{subfigure}[b]{0.45\textwidth}
        \includegraphics[width=\textwidth]{images/synthetic.png}
        \caption{Synthetic image, black on white.}
        \label{fig:syn1}
    \end{subfigure}
    ~ %add desired spacing between images, e. g. ~, \quad, \qquad, \hfill etc. 
      %(or a blank line to force the subfigure onto a new line)
    \begin{subfigure}[b]{0.45\textwidth}
        \includegraphics[width=\textwidth]{images/synthetic_2.png}
        \caption{Synthetic image, white on black.}
        \label{fig:syn2}
    \end{subfigure}
    ~ %add desired spacing between images, e. g. ~, \quad, \qquad, \hfill etc. 
    %(or a blank line to force the subfigure onto a new line)    
    \caption{Synthetic test images for edge detection algorithms. \subref{fig:syn1} shows various gray levels that require an adaptive algorithm. \subref{fig:syn2}
    shows more challenging edge detection tests that have crossing lines. Fusing these into full segments typically requires algorithms like the Hough transform.
    This is an example of using subfigures, with \texttt{subref}s in the caption.
    }\label{fig:synthetic}
\end{figure}

\clearpage

\subsection{Equations}

Equations should be typeset correctly and precisely. Make sure you get parenthesis sizing correct, and punctuate equations correctly 
(the comma is important and goes \textit{inside} the equation block). Explain any symbols used clearly if not defined earlier. 

For example, we might define:
\begin{equation}
    \hat{f}(\xi) = \frac{1}{2}\left[ \int_{-\infty}^{\infty} f(x) e^{2\pi i x \xi} \right],
\end{equation}    
where $\hat{f}(\xi)$ is the Fourier transform of the time domain signal $f(x)$.

\subsection{Algorithms}
Algorithms can be set using \texttt{algorithm2e}, as in Algorithm \ref{alg:metropolis}.

% NOTE: line ends are denoted by \; in algorithm2e
\begin{algorithm}
    \DontPrintSemicolon
    \KwData{$f_X(x)$, a probability density function returing the density at $x$.\; $\sigma$ a standard deviation specifying the spread of the proposal distribution.\;
    $x_0$, an initial starting condition.}
    \KwResult{$s=[x_1, x_2, \dots, x_n]$, $n$ samples approximately drawn from a distribution with PDF $f_X(x)$.}
    \Begin{
        $s \longleftarrow []$\;
        $p \longleftarrow f_X(x)$\;
        $i \longleftarrow 0$\;
        \While{$i < n$}
        {
            $x^\prime \longleftarrow \mathcal{N}(x, \sigma^2)$\;
            $p^\prime \longleftarrow f_X(x^\prime)$\;
            $a \longleftarrow \frac{p^\prime}{p}$\;
            $r \longleftarrow U(0,1)$\;
            \If{$r<a$}
            {
                $x \longleftarrow x^\prime$\;
                $p \longleftarrow f_X(x)$\;
                $i \longleftarrow i+1$\;
                append $x$ to $s$\;
            }
        }
    }
    
\caption{The Metropolis-Hastings MCMC algorithm for drawing samples from arbitrary probability distributions, 
specialised for normal proposal distributions $q(x^\prime|x) = \mathcal{N}(x, \sigma^2)$. The symmetry of the normal distribution means the acceptance rule takes the simplified form.}\label{alg:metropolis}
\end{algorithm}

\subsection{Tables}

If you need to include tables, like Table \ref{tab:operators}, use a tool like https://www.tablesgenerator.com/ to generate the table as it is
extremely tedious otherwise. 

\begin{table}[]
    \caption{The standard table of operators in Python, along with their functional equivalents from the \texttt{operator} package. Note that table
    captions go above the table, not below. Do not add additional rules/lines to tables. }\label{tab:operators}
    %\tt 
    \rowcolors{2}{}{gray!3}
    \begin{tabular}{@{}lll@{}}
    %\toprule
    \textbf{Operation}    & \textbf{Syntax}                & \textbf{Function}                            \\ %\midrule % optional rule for header
    Addition              & \texttt{a + b}                          & \texttt{add(a, b)}                                    \\
    Concatenation         & \texttt{seq1 + seq2}                    & \texttt{concat(seq1, seq2)}                           \\
    Containment Test      & \texttt{obj in seq}                     & \texttt{contains(seq, obj)}                           \\
    Division              & \texttt{a / b}                          & \texttt{div(a, b) }  \\
    Division              & \texttt{a / b}                          & \texttt{truediv(a, b) } \\
    Division              & \texttt{a // b}                         & \texttt{floordiv(a, b)}                               \\
    Bitwise And           & \texttt{a \& b}                         & \texttt{and\_(a, b)}                                  \\
    Bitwise Exclusive Or  & \texttt{a \textasciicircum b}           & \texttt{xor(a, b)}                                    \\
    Bitwise Inversion     & \texttt{$\sim$a}                        & \texttt{invert(a)}                                    \\
    Bitwise Or            & \texttt{a | b}                          & \texttt{or\_(a, b)}                                   \\
    Exponentiation        & \texttt{a ** b}                         & \texttt{pow(a, b)}                                    \\
    Identity              & \texttt{a is b}                         & \texttt{is\_(a, b)}                                   \\
    Identity              & \texttt{a is not b}                     & \texttt{is\_not(a, b)}                                \\
    Indexed Assignment    & \texttt{obj{[}k{]} = v}                 & \texttt{setitem(obj, k, v)}                           \\
    Indexed Deletion      & \texttt{del obj{[}k{]}}                 & \texttt{delitem(obj, k)}                              \\
    Indexing              & \texttt{obj{[}k{]}}                     & \texttt{getitem(obj, k)}                              \\
    Left Shift            & \texttt{a \textless{}\textless b}       & \texttt{lshift(a, b)}                                 \\
    Modulo                & \texttt{a \% b}                         & \texttt{mod(a, b)}                                    \\
    Multiplication        & \texttt{a * b}                          & \texttt{mul(a, b)}                                    \\
    Negation (Arithmetic) & \texttt{- a}                            & \texttt{neg(a)}                                       \\
    Negation (Logical)    & \texttt{not a}                          & \texttt{not\_(a)}                                     \\
    Positive              & \texttt{+ a}                            & \texttt{pos(a)}                                       \\
    Right Shift           & \texttt{a \textgreater{}\textgreater b} & \texttt{rshift(a, b)}                                 \\
    Sequence Repetition   & \texttt{seq * i}                        & \texttt{repeat(seq, i)}                               \\
    Slice Assignment      & \texttt{seq{[}i:j{]} = values}          & \texttt{setitem(seq, slice(i, j), values)}            \\
    Slice Deletion        & \texttt{del seq{[}i:j{]}}               & \texttt{delitem(seq, slice(i, j))}                    \\
    Slicing               & \texttt{seq{[}i:j{]}}                   & \texttt{getitem(seq, slice(i, j))}                    \\
    String Formatting     & \texttt{s \% obj}                       & \texttt{mod(s, obj)}                                  \\
    Subtraction           & \texttt{a - b}                          & \texttt{sub(a, b)}                                    \\
    Truth Test            & \texttt{obj}                            & \texttt{truth(obj)}                                   \\
    Ordering              & \texttt{a \textless b}                  & \texttt{lt(a, b)}                                     \\
    Ordering              & \texttt{a \textless{}= b}               & \texttt{le(a, b)}                                     \\
    % \bottomrule
    \end{tabular}
    \end{table}
\subsection{Code}

Avoid putting large blocks of code in the report (more than a page in one block, for example). Use syntax highlighting if possible, as in Listing \ref{lst:callahan}.

\begin{lstlisting}[language=python, float, caption={The algorithm for packing the $3\times 3$ outer-totalistic binary CA successor rule into a 
    $16\times 16\times 16\times 16$ 4 bit lookup table, running an equivalent, notionally 16-state $2\times 2$ CA.}, label=lst:callahan]
    def create_callahan_table(rule="b3s23"):
        """Generate the lookup table for the cells."""        
        s_table = np.zeros((16, 16, 16, 16), dtype=np.uint8)
        birth, survive = parse_rule(rule)

        # generate all 16 bit strings
        for iv in range(65536):
            bv = [(iv >> z) & 1 for z in range(16)]
            a, b, c, d, e, f, g, h, i, j, k, l, m, n, o, p = bv

            # compute next state of the inner 2x2
            nw = apply_rule(f, a, b, c, e, g, i, j, k)
            ne = apply_rule(g, b, c, d, f, h, j, k, l)
            sw = apply_rule(j, e, f, g, i, k, m, n, o)
            se = apply_rule(k, f, g, h, j, l, n, o, p)

            # compute the index of this 4x4
            nw_code = a | (b << 1) | (e << 2) | (f << 3)
            ne_code = c | (d << 1) | (g << 2) | (h << 3)
            sw_code = i | (j << 1) | (m << 2) | (n << 3)
            se_code = k | (l << 1) | (o << 2) | (p << 3)

            # compute the state for the 2x2
            next_code = nw | (ne << 1) | (sw << 2) | (se << 3)

            # get the 4x4 index, and write into the table
            s_table[nw_code, ne_code, sw_code, se_code] = next_code

        return s_table

\end{lstlisting}

%==================================================================================================================================
\chapter{Evaluation} 

TODOO:
\begin{itemize}
    \item Summarise based on results in user study
\end{itemize}

How good is your solution? How well did you solve the general problem, and what evidence do you have to support that?

\section{Guidance}
\begin{itemize}
    \item
        Ask specific questions that address the general problem.
    \item
        Answer them with precise evidence (graphs, numbers, statistical
        analysis, qualitative analysis).
    \item
        Be fair and be scientific.
    \item
        The key thing is to show that you know how to evaluate your work, not
        that your work is the most amazing product ever.
\end{itemize}

\section{Evidence}
Make sure you present your evidence well. Use appropriate visualisations, reporting techniques and statistical analysis, as appropriate.

If you visualise, follow the basic rules, as illustrated in Figure \ref{fig:boxplot}:
\begin{itemize}
\item Label everything correctly (axis, title, units).
\item Caption thoroughly.
\item Reference in text.
\item \textbf{Include appropriate display of uncertainty (e.g. error bars, Box plot)}
\item Minimize clutter.
\end{itemize}

See the file \texttt{guide\_to\_visualising.pdf} for further information and guidance.

\begin{figure}
    \centering
    \includegraphics[width=1.0\linewidth]{images/boxplot_finger_distance.pdf}    

    \caption{Average number of fingers detected by the touch sensor at different heights above the surface, averaged over all gestures. Dashed lines indicate
    the true number of fingers present. The Box plots include bootstrapped uncertainty notches for the median. It is clear that the device is biased toward 
    undercounting fingers, particularly at higher $z$ distances.
    }

    % use the notation fig:name to cross reference a figure
    \label{fig:boxplot} 
\end{figure}


%==================================================================================================================================
\chapter{Conclusion}    
Summarise the whole project for a lazy reader who didn't read the rest (e.g. a prize-awarding committee).
\section{Guidance}
\begin{itemize}
    \item
        Summarise briefly and fairly.
    \item
        You should be addressing the general problem you introduced in the
        Introduction.        
    \item
        Include summary of concrete results (``the new compiler ran 2x
        faster'')
    \item
        Indicate what future work could be done, but remember: \textbf{you
        won't get credit for things you haven't done}.
\end{itemize}

%==================================================================================================================================
%
% 
%==================================================================================================================================
%  APPENDICES  

\begin{appendices}

\chapter{Appendices}

Typical inclusions in the appendices are:

\begin{itemize}
\item
  Copies of ethics approvals (required if obtained)
\item
  Copies of questionnaires etc. used to gather data from subjects.
\item
  Extensive tables or figures that are too bulky to fit in the main body of
  the report, particularly ones that are repetitive and summarised in the body.

\item Outline of the source code (e.g. directory structure), or other architecture documentation like class diagrams.

\item User manuals, and any guides to starting/running the software.

\end{itemize}

\textbf{Don't include your source code in the appendices}. It will be
submitted separately.

\end{appendices}

%==================================================================================================================================
%   BIBLIOGRAPHY   

% The bibliography style is abbrvnat
% The bibliography always appears last, after the appendices.

\bibliographystyle{abbrvnat}

\bibliography{l4proj}

\end{document}
