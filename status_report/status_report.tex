\documentclass{article}

\title{Status Report}
\author{Robbie Tippen - 2403237t}
\date{15th December 2023}

\begin{document}
\maketitle

\section{Proposal}

\subsection{Motivation}

Imagine an inclusive initiative tailored specifically for individuals with intellectual disabilities, including conditions like Alzheimer's. Through Unity, we plan to create a captivating 3D world that generates music in response to exploration. This project is inspired by the unique relationship between music and cognitive function and its therapeutic potential for those with dementia. By immersing users in a visually engaging environment where music dynamically responds to movement and actions in the environment, we aim to provide a multisensory experience that not only captivates but also potentially aids cognitive well-being in an enjoyable, interactive manner.

\subsection{Aims}

This project aims to create an immersive gaming experience within a Unity 3D open-world environment. The primary focus is on implementing new instruments that unlock through minigames, offering players an interactive way to shape the in-game musical landscape. The in-game music will dynamically change in response to the player's movements, creating a unique sound experience linked to the player's actions. The project's effectiveness in captivating players and possibly enhancing cognitive well-being will be experimentally validated.

\subsection{Progress}

\begin{itemize}
  \item Researched project topic, focusing on "Music and dementia" readings.
  \item Name of the project decided, HarmonyScape, as well as the genre of game, 3D Exploration.
  \item Implemented 3D movement controls controlled with the keyboard.
  \item Implemented a 3rd Person Camera System controlled with the mouse.
  \item Built a terrain with peaks and valleys, signifying points of interest.
  \item Started a design document outlining the game's environment, back story, and gameplay mechanics.
  \item Added various instruments to the game that, upon collection, contribute to the base music and are displayed through icons in the HUD.
\end{itemize}

\subsection{Problems and Risks}

\subsubsection{Problems}

\begin{itemize}
  \item The current game version features placeholder music, presenting a notable shortfall in achieving the games intended purpose. There is a need to develop music incorporating all in-game instruments.
  \item Outlined minigames have not yet been implemented, and therefore cannot be tested.
  \item No clear direction for users. (No way of knowing where to go next)
\end{itemize}

\subsubsection{Risks}

\begin{itemize}
  \item Unclear how to evaluate the success of the project as unable to run user studies with intended user group. Mitigation: Do studies with users not specifically in the intended user group to gather insights on user experience in conjunction with research on previous examples that are similar, implementing functionality that mirrors needs previously gathered for the target user group.
  \item Limited access to quality music samples. Mitigation: Creating the music myself so it is fit for purpose, incorporating all the instruments.
  \item There is a potential challenge in implementing the movement-based music interaction feature, as it has not been integrated into the project at this stage. Mitigation: The project will allocate significant focus on the development of the movement-based music interaction over the next week and during the initial two weeks of semester 2. An assessment will be conducted during this period, leading to a decision on whether to include this feature in the final implementation based on its feasibility.
\end{itemize}

\subsection{Plan}

\begin{itemize}
  \item Week 1-2: Develop movement-based music interaction. \\
    \textbf{Deliverable:} Complete implementation of the interaction with the music based on player interaction.
  \item Week 3-5: Implement minigames to unlock instruments. \\
    \textbf{Deliverable:} Complete implementation of the 'minigame' system that unlocks instruments to be added to the games' music.
  \item Week 6: Research on how to best evaluate performance of final system. \\
    \textbf{Deliverable:} Detailed evaluation plan, with participant numbers, information sheet and analysis plan.
  \item Week 7-9: Final implementation and improvements to user direction. \\
    \textbf{Deliverable:} Polished software ready, implementation of user direction, ready for evaluation stage.
  \item Week 9: Evaluation experiments run. \\
    \textbf{Deliverable:} Quantitative measures of usability and qualitative measures of effectiveness for at least ten users.
  \item Week 8-10: Write up. \\
    \textbf{Deliverable:} First draft submitted to supervisor two weeks before final deadline.
\end{itemize}

\end{document}
